\documentclass[11pt,a4paper]{article}
\usepackage[left=2cm, top=2.5 cm, text={17cm, 24cm}]{geometry}
\usepackage[czech]{babel}
\usepackage[utf8]{inputenc}
\usepackage{times}
\usepackage{amsmath, amsthm, amssymb}
\usepackage{multirow}
\usepackage[ruled,czech,linesnumbered,longend,noline]{algorithm2e}
\usepackage{algorithmic}
\usepackage{picture}
\usepackage{graphics}
\usepackage{lscape}

\newcommand{\myuv}[1]{\quotedblbase #1\textquotedblleft}

\begin{document}

\begin{titlepage}
\begin{center}
\textsc{\Huge Vysoké učení technické v Brně\\ \medskip
\huge Fakulta informačních technologií}\\[30mm]
	
\begin{center}
\scalebox{1} {\includegraphics{fit_logo.eps}}
\end{center}
\vspace{\stretch{0.3}}

\Huge{Implementácia interpretu imperatívneho jazyka}\\[4mm]
\Huge{IFJ16}\\[2mm]
\LARGE{Tím 052, variant a/2/II}\\
\LARGE{Rozšírenia: }\\

\vspace{\stretch{0.8}}

\begin{flushright}
\noindent \underline{Jakub Semrič}\\
Petr Rusiňák\\
Kryštof Rykala\\
Martin Mikan\\
\today \hfill         Martin Polakovič \newpage
\end{flushright}

\end{center}

\pagebreak
\thispagestyle{empty}
\tableofcontents
\pagebreak

\end{titlepage}
\newpage

\section{Úvod}

Tento dokument popisuje návrh a implementáciu interpretu imperatívneho jazyka IFJ16, ktorý je podmnožinou jazyka JAVA.
Program funguje ako konzolová aplikácia, ktorá načíta zdrojový súbor programu v~jazyku IFJ16 a následne ho interpretuje. V~prípade výskytu chyby, program vracia kód chyby.

Dokument sa skladá z~viacerých častí. V~kapitole \ref{popis} je popísané riešenie jednotlivých častí interpretu, v~kapitole \ref{algoritmy} sa venujeme použitým algoritmom a v~nasledujúcej kapitole \ref{struktury} dátovým štruktúram. Kapitola \ref{rozsirenia} dokumentuje vypracované rozšírenia, kapitola \ref{testovanie} rozoberá náš postup pri testovaní a kapitola \ref{tim} je zameraná na zhodnotenie našej práce v~tíme. V~poslednej kapitole \ref{zaver} je krátke shrnutie.

\section{Popis riešenia a implementácia jazyka IFJ16} 
\label{popis}

Projekt sme implementovali z niekoľkých funkčných celkov. V nasledujúcich podkapitolách sú jednotlivé moduly detailne popísané.

    \subsection{Lexikálny analyzátor}
    \label{lexer}

    TODO

    \subsection{Syntaktický a sémantický analyzátor}
    \label{parser}

    TODO

        \subsubsection{Spracovanie jazykových inštrukcií - rekurzívny zostup}
        \label{rekurzia}
    
        TODO
    
        \subsubsection{Precedenčná analýza výrazov}
        \label{precedencna analyza}
    
    TODO
    
    \subsection{Interpret vnútorného kódu}
    \label{interpret}
    
    TODO

\section{Použité algoritmy} 
\label{algoritmy}

TODO

    \subsection{Radiaci algoritmus - Heap Sort}
    
    TODO

    \subsection{Vyhľadávací algoritmus - Knuth-Morris-Pratt}
    
    TODO
    
\section{Využívané dátové štruktúry} 
\label{struktury}

        \subsection{Tabuľka s rozptýlenými položkami}
    
    TODO
    
        \subsection{Zásobník}
    
    TODO
    
        \subsection{Reťazec}
    
    TODO
    
\section{Rozšírenia} 
\label{rozsirenia}

    \subsection{SIMPLE}
    
    TODO
    
    \subsection{BOOLOP}
    
    TODO
    
\section{Testovanie} 
\label{testovanie}

TODO

\section{Práca v tíme} 
\label{tim}

    \subsection{Rozdelenie úloh}
    
    TODO doplnim tabulku

\section{Záverom} 
\label{zaver}

\section{Použité zdroje}

TODO - prosím pridajte link alebo názov zdroja, dohľadám bibliografické informácie

\section{Prílohy}


\subsection{Konečný automat}

\subsection{LL-gramatika}

\subsection{Precedenčná tabuľka}

\end{document}