\documentclass[11pt,a4paper]{article}
\usepackage[left=2cm, top=2.5 cm, text={17cm, 24cm}]{geometry}
\usepackage[czech]{babel}
\usepackage[utf8]{inputenc}
\usepackage{times}
\usepackage{amsmath, amsthm, amssymb}
\usepackage{multirow}
\usepackage[ruled,czech,linesnumbered,longend,noline]{algorithm2e}
\usepackage{algorithmic}
\usepackage{picture}
\usepackage{graphics}
\usepackage{lscape}

\newcommand{\myuv}[1]{\quotedblbase #1\textquotedblleft}

\begin{document}

\begin{titlepage}
\begin{center}
\textsc{\Huge Vysoké učení technické v Brně\\ \medskip
\huge Fakulta informačních technologií}\\[30mm]
	
\begin{center}
\scalebox{1} {\includegraphics{fit_logo.eps}}
\end{center}
\vspace{\stretch{0.3}}

\Huge{Implementácia interpretu imperatívneho jazyka}\\[4mm]
\Huge{IFJ16}\\[2mm]
\LARGE{Tím 052, variant a/2/II}\\
\LARGE{Rozšírenia: }\\

\vspace{\stretch{0.8}}

\begin{flushright}
\noindent \underline{Jakub Semrič}\\
Petr Rusiňák\\
Kryštof Rykala\\
Martin Mikan\\
\today \hfill         Martin Polakovič \newpage
\end{flushright}

\end{center}

\pagebreak
\thispagestyle{empty}
\tableofcontents
\pagebreak

\end{titlepage}
\newpage

\section{Úvod}

Tento dokument popisuje návrh a implementáciu interpretu imperatívneho jazyka IFJ16, ktorý je podmnožinou jazyka JAVA.
Program funguje ako konzolová aplikácia, ktorá načíta zdrojový súbor programu v~jazyku IFJ16 a následne ho interpretuje. V~prípade výskytu chyby, program vracia kód chyby.

Dokument sa skladá z~viacerých častí. V~kapitole \ref{tim} je stručne popísaný vývoj projektu, v časti \ref{popis} je popísané riešenie jednotlivých častí interpretu, v~kapitole \ref{algoritmy} sa venujeme použitým algoritmom a v~nasledujúcej kapitole \ref{struktury} dátovým štruktúram. Kapitola \ref{rozsirenia} dokumentuje vypracované rozšírenia, kapitola \ref{testovanie} rozoberá náš postup pri testovaní interpretu.


\section{Práca v tíme} 
\label{tim}

    \subsection{Rozdelenie úloh}
    \begin{center}
    \begin{tabular}{|c|c|}
    \hline
    Lexikálny analyzátor & Kryštof Rykala \\
    \hline
    Parser & Jakub Semrič \\
    \hline
    Precedenčná analýza & Peter Rusiňák \\
    \hline
    Algoritmy, tabuľka symbolov & Martin Polakovič \\
    \hline
    Interpret & Martin Mikan \\
    \hline
    \end{tabular}
\end{center}
    
    \subsection{Priebeh vývoja}
    \label{vyvoj}
    Pre nedostatok vedomostí týmu a nestabilitu návrhu, vývoj prebiehal v niekoľkých iteráciach. Na začiatku každej iterácie
    sa plánovalo s primeranou mierou, v ďalšej fáze sa implementovalo a na konci iterácie sa zamýšľalo čo sa bude robiť ďalej a
    ako riešiť novo vzniknuté problémy.

    Pri návrhu a implementácii
    jednotlivých modulov sa postupovalo tak, aby výsledok bol čo najviac flexibilný a~interoperabilný s~ostatnými modulmi.
    Na každom stretnutí, ktoré
    bolo približne raz za tri týždňe, sa stanovili ciele, ktoré museli byť do najbližšieho stretnutia dosiahnuté. Tieto ciele
    sa naozaj podarilo za určený čas splniť a spolu zo zavedením regresných testov vývoj celého projektu prebiehal  prekvapujúco
    rýchlo.

\section{Popis riešenia a implementácia jazyka IFJ16}
\label{popis}

Projekt sme implementovali z niekoľkých funkčných celkov. V nasledujúcich podkapitolách sú jednotlivé moduly detailne popísané.

    \subsection{Lexikálny analyzátor}
    \label{lexer}
    Lexikálny analyzátor tvorí vstupnú časť interpretu. Model tvorí deterministický konečný automat.
    Spracováva zdrojový súbor a~identifikuje v ňom platńe lexémy, ktoré posiela syntaktickému analyzátoru vo forme tokenu.
    Token obsahuje typ lexémy a~v~prípade aj hodnotu ak sa jedná o~celočíselný, desatinný alebo reťazcový literál.

    Konečný automat je implementovaný tak aby vedel spracovat aj jednoduché \texttt{//} a~blokové \texttt{/* */} komentáre.
    Po ukončení komentáru  automat prejde do začínajúceho stavu a~začne spracovávať nasledujúcu lexému alebo ďalší komentár.

    Model konečného automatu môžete násjť v~prílohe 9.1. % TODO referencia na automat


    \subsection{Syntaktický a sémantický analyzátor}
    \label{parser}
    Syntaktický analýzátor (parser) je hlavná časť interpretu. Parser dostáva riadenie hned po otvorení súboru. Má za úlohu nielen
    preveriť syntax a~sémantiku zdrojového súboru, ale taktiež vkladá inštrukcie do inštrukčnej pásky. Jeho implementácia spočíva
    v~rekurzívnom zostupe, ktorý je riadený LL(1) gramatikou. % TODO referencia na gramatiku

    Keďže jazyk \emph{IFJ16} je podmnožinou jazyka \emph{JAVA SE 8}, ktorého preklad je zložený z dvoch priechodov, je parser zložený
    taktiež z~dvoch priechodov. Prvý priechod analyzuje program zo syntaktického a~lexikálneho hľadiska, súčastne vkladá do tabuľky
    symbolov názvy tried, statických funkcií a~premmenných a kontroluje pokus o redefiníciu triedy, funkcie alebo statickej premmennej.
    Druhý priechod kontroluje semántiku programu, spracováva telá funkcií a~vytvára inštrukčnú pásku.



        \subsubsection{Spracovanie jazykových inštrukcií\,-- rekurzívny zostup}
        \label{rekurzia}
        Rekurzívny zostup riadený LL(1) gramatikou, bol implementovaný tak, že každý neterminál v~gramatike bol reprezentovaný funkciou,
        z~ktorej sa postupne volali ďalšie funkcie na základe aktuálneho tokenu a~pravidla v~gramatike. Aktuálne načítaný token zase reprezentoval
        terminálne symboly gramatiky.

        Druhý priechod je tiež založený na rekurzívnom zostupe, avšak úloha jednotlivých funkcií sa od prvého priechodu značne odlišuje.
        Pri inicializácii statických premenných sa výraz za \texttt{=} vyhodnotí precedenčnou analýzou a~vložia sa inštrukcie do hlavného
        inštrukčného listu. Pri analýze funkcií sa postupne vložia ďalšie inštrukcie do inštrukčného listu aktuálne spracovávanej funkcie
        a~v~prípade vyhodnotenia výrazu (pri príkazoch
        \texttt{while/if/else if ( \emph{EXP} ), id = \emph{EXP}}) sa predá riadenie precedenčnej analýze.

        V~prípade volania funkcie sa skontroluje existencia funkcie v~členskej triede, počet a~typ parametrov a~v~prípade priradenia
        návratový typ funkcie a~premennej, do ktorej je volanie priradené.
        Do inštrukčného listu sa vloží inštrukcia \texttt{CALL}, ktorá uloží na hlavný zásobnik postupne odkaz na parametre
        volanej funkcie a~inštrukcia \texttt{LAB} iba určí návratový bod, ktorý bude použitý pri skoku z~volanej funkcie.

        Pokiaľ nebola nájdená syntaktická alebo sémantická chyba, po narazení na token typu koniec súboru sa v~tabuľke symbolov nájde
        trieda \emph{Main} a~funkcia zapuzdrená v~tejto triede \emph{run()} a~spojí sa hlavný inštrukčný list s~inštrukčným listom funkcie
        \emph{run()}. Pokiaľ požadovane entity boli nájdené analýza končí a riadenie je predané interpretu.

        \subsubsection{Precedenčná analýza výrazov}
        \label{precedencna analyza}
        % TODO referencia na tabulku 
    TODO
    
    \subsection{Interpret vnútorného kódu}
    \label{interpret}
    Interpret postupne vykonáva inštrukcie v~inštrukčnom liste až kým nenarazí na chybu (napr. práca s neinicializovanou premennou)
    alebo na koniec listu. Každá inštrukcia má osobitnú implementáciu. Interpret aktívne využíva hlavný zásobník a~zásobník rámcov.
    Lokálne premenné s~ktorými má pracovať  sú v~tabuľke, ktorá je na vrchole zásobníku rámcov. Statické premenné a~konštanty sú spracovávané
    priamo bez vyhľadania v tabuľke.

       \subsubsection{Volanie a návrat z funkcií}
       \label{funkcia}
       Pri inštrukcii volania funkcie \texttt{CALL} sa podľa prvej adresy (3-adresný kód inštrukcie) zistí o~akú funkciu sa jedná
       a~koľko parametrov je na zásobníku.
       Vytvorí sa nový rámec s lokálnou tabuľkou skopírovanou z lokálnej tabuľky volanej funkcie a súčastne sa naplnia parametre
       hodnotamy premenných na zásobníku. Rámec sa uloží na vrchol zásobníku rámcov, potom
       nasleduje skok na prvú položku inštrukčného listu danej funkcie.
       Pri inštrukcii \texttt{RET} (príkaz \texttt{return}) nasleduje skok na miesto odkiaľ bola funkcia volaná, takže na inštrukciu za
       inštrukciou \texttt{CALL} v predošlom liste. Miesto reprezentované inštrukciou \texttt{LAB} je uložené na hlavný zásobník aby sa
       vedelo, ktorou inštruciou sa má pokračovať. Návratová hodnota funkcie je uložená v~špeciálnych premenných uložených v~globálnej
       tabuľke symbolov. Jedná sa o emuláciu registrov, ktoré zjednodušujú implementáciu celého interpretu.

    
    

\section{Použité algoritmy} 
\label{algoritmy}

TODO

    \subsection{Radiaci algoritmus\,-- Heap Sort}
    
    TODO

    \subsection{Vyhľadávací algoritmus\,-- Knuth-Morris-Pratt}
    
    TODO
    
\section{Využívané dátové štruktúry} 
\label{struktury}

        \subsection{Tabuľka s rozptýlenými položkami}
        Tabuľka s rozptýlenými položkami je implementovaná ako pole jednosmerne viazaných zoznamov symbolov. Symbol obsahuje
        mimo identifikátora aj ukazatel na členskú triedu. Vyhľadávanie spočíva v nájdení symbolu s~rovnakým identifikátorom
        a súčastne rovnakou členskou triedou. Toto vyhľadávanie zabráňuje kolíziam symbolov rovnakých mien v rôznych úrovniach.
        Kedže lokálne premenné nemajú členskú triedu existuje varianta vyhľadávaní v tabuľke iba s jedným kľúčom.

        V interprete sa používaju dva typy tabuliek: lokálna a~globálna. V~globálnej tabuľke sú uložené symboly reprezentujúce
        triedy, funkcie, konštanty, statické premenné a~registre. V~lokálnej tabuľke, sú uložené iba lokálne premenné a~parametre
        funkcií, ktoré majú vlastnú tabuľku symbolov.

        \subsection{Zásobník}
        Zásobník je implementovaný ako pole ukazateľov určitej kapacity. Ak je zásobník plný dôjde
        k automatickej realokácii a zvýšeniu kapacity zásobníku. V~interprete sa používaju dva zásobníky. Jeden z~nich
        sa používa na uloženie odkazov na miesta skokov a odkazy na parametre funkcií. Druhý zásobník je použitý na ukladanie
        rámcov, ktoré obsahujú lokálnu tabuľku symbolov.

\section{Rozšírenia} 
\label{rozsirenia}

    \subsection{SIMPLE}
    Rozšírenie \emph{SIMPLE} umožňuje skrátený zápis konštrukcií cyklov a podmienok vynechaním zložených zátvoriek. Keďže by terminálne symboly
    \texttt{\{ \} } moholi byť vynechané, jednalo by sa už o kontextovú gramatiku. Tento problém bol vyriešený implementačne bez zmeny
    gramatiky tak, že sa v prípade absencie zložených zátvoriek namiesto volania funkcie reprezentujúcej neterminál \texttt{STLIST}, vložila
    funkcia reprezentujúca neterminál \texttt{STAT}, čiže sa namiesto zloženého príkazu vykonal jednoduchý príkaz.

    \subsection{BOOLOP}
    Rozšírenie umožňuje poporovať nový dátový typ\,-- \emph{boolean}. Interpret podporuje deklarácie premenných a~funkcií typu boolean. Navyše
    umožnuje vyhodnocovanie zložených podmienok.

\section{Testovanie} 
\label{testovanie}
Návrh testov bol založený na metóde ekvivalenčných tried. Testy mali charakter regresných testov, takže zmena zdrojových súborov nebola možná ak
všetky testy neboli úspešné. Testy sa delili do dvoch skupín\,-- testy detekcie chýb a interpretačné testy. Pri testovaní bol použitý program
\texttt{gcovr}, ktrorý nám ukázal, ktoré časti interpretu neboli otestované.
% TODO odkaz na gcovr: http://gcovr.com/

\section{Použité zdroje}

TODO - prosím pridajte link alebo názov zdroja, dohľadám bibliografické informácie

\newpage
\section{Prílohy}


\subsection{Konečný automat}


\newpage

\subsection{LL-gramatika}
%TODO 2 stlpce
PROG $\rightarrow$ BODY eof \\
BODY $\rightarrow$ CLASS BODY \\
BODY $\rightarrow$ $\epsilon$ \\
CLASS $\rightarrow$ class id \{ CBODY \} \\
CBODY $\rightarrow$ static TYPE id CBODY2 CBODY \\
CBODY $\rightarrow$ $\epsilon$ \\
CBODY2 $\rightarrow$ INIT ; \\
CBODY2 $\rightarrow$ FUNC \\
FUNC $\rightarrow$ ( PAR ) FBODY \\
FBODY $\rightarrow$ \{ STLIST \} \\
PAR $\rightarrow$ TYPE PAR2 \\
PAR $\rightarrow$ $\epsilon$ \\
PAR2 $\rightarrow$ $\epsilon$ \\
PAR2 $\rightarrow$ id PAR3 \\
PAR3 $\rightarrow$ com TYPE id PAR3 \\
PAR3 $\rightarrow$ $\epsilon$ \\
STLIST $\rightarrow$ STAT STLIST \\
STLIST $\rightarrow$  $\epsilon$ \\
STAT $\rightarrow$ while ( EXPR ) \{ STLIST \} \\
STAT $\rightarrow$ if ( EXPR ) \{ STLIST \} ELSE \\
STAT $\rightarrow$ return RET ; \\
STAT $\rightarrow$ id = CALLEXPR ; \\
STAT $\rightarrow$ fullid = CALLEXPR ; \\
STAT $\rightarrow$ static TYPE id INIT ; \\
STAT $\rightarrow$ TYPE id INIT ; \\
CALLEXPR $\rightarrow$ fullid ( ARG ) \\
CALLEXPR $\rightarrow$ id ( ARG ) \\
CALLEXPR $\rightarrow$ EXPR \\
ARG $\rightarrow$ $\epsilon$ \\
ARG $\rightarrow$ id ARG2 \\
ARG $\rightarrow$ fullid ARG2 \\
ARG2 $\rightarrow$ $\epsilon$ \\
ARG2 $\rightarrow$ com ARG3 ARG2 \\
ARG3 $\rightarrow$ fullid \\
ARG3 $\rightarrow$ id \\
INIT $\rightarrow$ = CALLEXPR \\
INIT $\rightarrow$ $\epsilon$ \\
RET $\rightarrow$ EXPR \\
RET $\rightarrow$ $\epsilon$ \\
ELSE $\rightarrow$ $\epsilon$ \\
ELSE $\rightarrow$ else ELSE2 \\
ELSE2 $\rightarrow$ if ( EXPR ) \{ STLIST \} ELSE \\
ELSE2 $\rightarrow$ \{ STLIST \} \\
TYPE $\rightarrow$ void \\
TYPE $\rightarrow$ int \\
TYPE $\rightarrow$ string \\
TYPE $\rightarrow$ double \\
TYPE $\rightarrow$ boolean \\

\newpage
\subsection{Precedenčná tabuľka}

\end{document}
